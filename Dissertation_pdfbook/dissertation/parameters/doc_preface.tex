
%% This sets the page style and numbering for preliminary sections.
\begin{preliminary}

%% This generates the title page from the information given in preamble.tex.
\maketitle
%\addcontentsline{toc}{chapter}{Certificate of Examination}
%\makecert
\newpage


%\addcontentsline{toc}{chapter}{Co-Authorship Statement}
%\coauthor{\makecoauthor}  %comment this out if none
%\newpage
%\addcontentsline{toc}{chapter}{Acknowlegements}
%\acknowlege{\makeacknowlege}	%as above
\setcounter{page}{2}
\addcontentsline{toc}{chapter}{Abstract}
\Large\begin{center}ABSTRACT\end{center}\normalsize

\begin{center}
	\singlespacing
\large ESSAYS ON MEASURING CREDIT AND PROPERTY PRICES GAPS
\end{center}

\doublespacing
\begin{center}
by\\
\end{center}

\begin{center}
\authorname\\
\end{center}

\singlespacing
\begin{center}
The University of Wisconsin-Milwaukee, 2022\\
Under the Supervision of Professor Kundan Kishor\\
\end{center}

\doublespacing
%%  ***  Put your Abstract here.   ***
%% (150 words for M.Sc. and 350 words for Ph.D.)
%\input{abstract}

My dissertation studies credit expansion and its effect on house prices and financial stability. In the first chapter, I examine the idea that house prices and credit to households are jointly determined, affecting each other in the short and long run. I decompose the movements of the two variables of interest into permanent long-run and transitory short-run components using an unobserved components vector autoregressive model. The dynamic model shows findings to support the hypothesis that a short-run positive shock to house prices is associated with an increase in household credit above its long-run trend. Furthermore, by utilizing additional information generated by the unobserved component model, our multivariate model performs better than univariate models in capturing the dynamics of household credit and house prices over the last three decades, especially during the recent financial crisis. I also estimated the predictive ability of cyclical components of a variable on their counterparts from other variables by employing cross-correlation coefficients in the VAR model.

In the second chapter, I propose a new method to measure the credit gap - the deviation of the credit-to-GDP ratio from its long-run trend. Here, I utilize the idea proposed in Nelson (2008) that the deviation of a non-stationary variable from its long-run trend should predict future changes in the variable. Since different trend-cycle decomposition methods of credit-to-GDP ratio provide different credit gap measures, I handle the model uncertainty by assigning weights to these different credit gap measures based on their relative out-of-sample predictive power based on Bates and Granger (1969) forecast combination method. I apply this approach to estimate the UK and the US credit gap using credit-to-GDP ratio data from 1960-2020.

My proposed credit gap measure dominates the alternate credit gaps, including the one provided by the Bank of International Settlements (BIS) regarding its relative out-of-sample predictive power. The proposed gap also has superior features in terms of early detection of turning points and relative insensitivity to the endpoint problem.

The third chapter of my dissertation overcomes model uncertainty in using the credit gap as an early warning indicator (EWI) of systemic financial crises in a binary outcome setting. I propose using model averaging of different credit gap measurements to achieve a better averaged model fit and out-of-sample prediction. I use binary logistic regression in a panel setup consisting of 40 countries. In this paper, I also propose a novel, superior criteria to judge the performance of an EWI than the one currently popularly used in the literature. The empirical results showed that our Bayesian averaged model could synthesize a single credit gap that outperforms other popularly studied credit gap measurements in terms of an early warning indicator.
\newpage


% \clearpage
% 
% \thispagestyle{empty}
% \null\vfill
% {
% 	\settowidth\longest{\Large\itshape Planning to write is not writing. Outlining, researching,}
% 	\centering
% 	\hspace{3cm}\parbox{\longest}{%
% 		\raggedright{\Large\itshape%
%       Planning to write is not writing. Outlining, researching, talking to people about what you're doing, none of that is writing. Writing is writing.\par\bigskip
% 		}   
% 		\raggedleft\MakeUppercase{E. L. Doctorow}\par%
% 		\raggedleft\MakeUppercase{\textit{Some book}}\par%
% }}
% 
% \vfill\vfill
% 
% \newpage


%%%%%%%%%%%%%%%%%%%%%%%%%%%%%%%%%%%%%%%%%%
%% %%%%%%%%  COPYRIGHT  PAGE  %%%%%%%%%%%%%%
%%%%%%%%%%%%%%%%%%%%%%%%%%%%%%%%%%%%%%%%%%

% Centering commands.
\null\vfill
\begin{center}
\begin{singlespace}
\copyright \ Copyright by Nam Nguyen Tam Hoai, 2022\\
All Rights Reserved
\end{singlespace}
\end{center}
\vfill\null
\newpage

%%%%%%%%%%%%%%%%%%%%%%%%%%%%%%%%%%%%%%%%%%
%% %%%%%%%%  DEDICATION  PAGE  %%%%%%%%%%%%%%
%%%%%%%%%%%%%%%%%%%%%%%%%%%%%%%%%%%%%%%%%%


% Centering commands.
\null\vfill
\begin{center}
	\begin{doublespace}
		To \\
		my parents,\\
		and my teachers,\\
	\end{doublespace}
\end{center}
\vfill\null
\newpage

%%%%%%%%%%%%%%%%%%%%%%%%%%%%%%%%%%%%%%%%%%%%%%%%%%%%
%%%%%%%%%%%%%%%% Table of contents %%%%%%%%%%%%%%%%%
% Source: UWO thesis template
\singlespacing
\setcounter{tocdepth}{1}

%https://latex-tutorial.com/tutorials/table-of-contents/
\renewcommand{\contentsname}{\centerline{TABLE OF CONTENTS}}
%\patchcmd{\tableofcontents}{\@starttoc}{\vspace{-1cm}\@starttoc}{}{}

%\renewcommand{\appendixpagename}{Appendix}
%\renewcommand{\appendixtocname}{Appendix}
%\renewcommand{\printtoctitle}[1]{\centering \Huge\sffamily TABLE OF CONTENTS}
\tableofcontents\newpage
\let\clearpage\relax

\newpage

\renewcommand{\listfigurename}{\centerline{LIST OF FIGURES}}
\addcontentsline{toc}{chapter}{List of Figures}
\listoffigures
\newpage
\let\clearpage\relax

\newpage

%\addcontentsline{toc}{chapter}{List of Algorithms}
\renewcommand{\listtablename}{\centerline{LIST OF TABLES}}
\addcontentsline{toc}{chapter}{List of Tables}
\listoftables
\newpage
\let\clearpage\relax

\newpage

% \addcontentsline{toc}{chapter}{List of Abbreviations, Symbols, and Nomenclature}
% \Large \textbf{List of Abbreviations, Symbols, and Nomenclature} \normalsize
% 
% You can define nomenclature from the text by using printnomenclature:
% \printnomenclature
% 
% You can also manually specify them:
% 
% \begin{tabular}{lcl}
% \\
% \multicolumn{3}{l}{\textbf{Terminology}}\\
% CNR & & Contrast-to-noise ratio \\
% SNR & \, & Signal-to-noise ratio \\
% \\
% \multicolumn{3}{l}{\textbf{Terms Introduced by this Thesis}}\\
% $\mathbb{ABC}$ && ABC \\
% DEF & \, & DEF \\
% \end{tabular}
% 
% \newpage

% \clearpage
% 
% \addcontentsline{toc}{chapter}{Preface}
% \Large\begin{center}\textbf{Preface}\end{center}\normalsize
% \onehalfspacing
% %\input{preface}
% 
% \newpage

%%%%%%%%%%%%%%%%%%%%%%%%%%%%%%%%%%%%%%%%%%%%%%%%%%%%
%%%%%%%%%%%%%%%% Table of contents %%%%%%%%%%%%%%%%%
% Source: LDM's dissertation toolkit
% 
% \begin{singlespace}
% \tableofcontents
% \end{singlespace}
% 
% \clearpage
% \addcontentsline{toc}{chapter}{\listtablename}
% \listoftables
% 
% \clearpage
% \addcontentsline{toc}{chapter}{\listfigurename}
% \listoffigures
% 
% \clearpage
% \addcontentsline{toc}{chapter}{\nomname}
% \printnomenclature
% 
% \clearpage
% \normalsize
% \pagenumbering{arabic}
% \setcounter{page}{1}

\end{preliminary}
%% End of the preliminary sections







