% Options for packages loaded elsewhere
\PassOptionsToPackage{unicode}{hyperref}
\PassOptionsToPackage{hyphens}{url}
%
\documentclass[
  ignorenonframetext,
]{beamer}
\usepackage{pgfpages}
\setbeamertemplate{caption}[numbered]
\setbeamertemplate{caption label separator}{: }
\setbeamercolor{caption name}{fg=normal text.fg}
\beamertemplatenavigationsymbolsempty
% Prevent slide breaks in the middle of a paragraph
\widowpenalties 1 10000
\raggedbottom
\setbeamertemplate{part page}{
  \centering
  \begin{beamercolorbox}[sep=16pt,center]{part title}
    \usebeamerfont{part title}\insertpart\par
  \end{beamercolorbox}
}
\setbeamertemplate{section page}{
  \centering
  \begin{beamercolorbox}[sep=12pt,center]{part title}
    \usebeamerfont{section title}\insertsection\par
  \end{beamercolorbox}
}
\setbeamertemplate{subsection page}{
  \centering
  \begin{beamercolorbox}[sep=8pt,center]{part title}
    \usebeamerfont{subsection title}\insertsubsection\par
  \end{beamercolorbox}
}
\AtBeginPart{
  \frame{\partpage}
}
\AtBeginSection{
  \ifbibliography
  \else
    \frame{\sectionpage}
  \fi
}
\AtBeginSubsection{
  \frame{\subsectionpage}
}
\usepackage{amsmath,amssymb}
\usepackage{lmodern}
\usepackage{iftex}
\ifPDFTeX
  \usepackage[T1]{fontenc}
  \usepackage[utf8]{inputenc}
  \usepackage{textcomp} % provide euro and other symbols
\else % if luatex or xetex
  \usepackage{unicode-math}
  \defaultfontfeatures{Scale=MatchLowercase}
  \defaultfontfeatures[\rmfamily]{Ligatures=TeX,Scale=1}
\fi
% Use upquote if available, for straight quotes in verbatim environments
\IfFileExists{upquote.sty}{\usepackage{upquote}}{}
\IfFileExists{microtype.sty}{% use microtype if available
  \usepackage[]{microtype}
  \UseMicrotypeSet[protrusion]{basicmath} % disable protrusion for tt fonts
}{}
\makeatletter
\@ifundefined{KOMAClassName}{% if non-KOMA class
  \IfFileExists{parskip.sty}{%
    \usepackage{parskip}
  }{% else
    \setlength{\parindent}{0pt}
    \setlength{\parskip}{6pt plus 2pt minus 1pt}}
}{% if KOMA class
  \KOMAoptions{parskip=half}}
\makeatother
\usepackage{xcolor}
\newif\ifbibliography
\usepackage{graphicx}
\makeatletter
\def\maxwidth{\ifdim\Gin@nat@width>\linewidth\linewidth\else\Gin@nat@width\fi}
\def\maxheight{\ifdim\Gin@nat@height>\textheight\textheight\else\Gin@nat@height\fi}
\makeatother
% Scale images if necessary, so that they will not overflow the page
% margins by default, and it is still possible to overwrite the defaults
% using explicit options in \includegraphics[width, height, ...]{}
\setkeys{Gin}{width=\maxwidth,height=\maxheight,keepaspectratio}
% Set default figure placement to htbp
\makeatletter
\def\fps@figure{htbp}
\makeatother
\setlength{\emergencystretch}{3em} % prevent overfull lines
\providecommand{\tightlist}{%
  \setlength{\itemsep}{0pt}\setlength{\parskip}{0pt}}
\setcounter{secnumdepth}{-\maxdimen} % remove section numbering
\PassOptionsToPackage{table}{xcolor}

\setbeamertemplate{caption}[numbered]

\setbeamertemplate{footline}{%
	\leavevmode%
	\hbox{\begin{beamercolorbox}[wd=.3\paperwidth,ht=2.5ex,dp=1.125ex,leftskip=.3cm plus1fill,rightskip=.3cm]{author in head/foot}%
			\usebeamerfont{author in head/foot}\insertshortauthor~(\insertshortinstitute)
		\end{beamercolorbox}%
		\begin{beamercolorbox}[wd=.7\paperwidth,ht=2.5ex,dp=1.125ex,leftskip=0cm plus1fill,rightskip=.3cm]{title in head/foot}%
			\usebeamerfont{title in head/foot}\insertshorttitle\;\;\;\;\hfill\insertframenumber\,/\,\inserttotalframenumber
	\end{beamercolorbox}}%
	\vskip0pt%
}

% \setbeamertemplate{footline}{%
% 	\leavevmode%
% 	\hbox{\begin{beamercolorbox}[wd=0.2\paperwidth,ht=2.5ex,dp=1.125ex,leftskip=.3cm plus1fill,rightskip=.3cm]{author in head/foot}%
% 			\usebeamerfont{author in head/foot}\insertshortauthor~(\insertshortinstitute)
% 		\end{beamercolorbox}%
% 		\begin{beamercolorbox}[wd=0.35\paperwidth,ht=2.5ex,dp=1.125ex,leftskip=.3cm,rightskip=.3cm plus1fil]{title in head/foot}%
% 			\usebeamerfont{title in head/foot}\insertshorttitle
% 		\end{beamercolorbox}%
% 		\begin{beamercolorbox}[wd=0.45\paperwidth,ht=2.5ex,dp=1.125ex,leftskip=.3cm,rightskip=.3cm plus1fil]{section in head/foot}%
% 			\usebeamerfont{section in head/foot}\insertsection \; insertframenumber\,/\,\inserttotalframenumber
% 		\end{beamercolorbox}}%
% 	\vskip0pt%
% }

%\AtBeginSection{}
\AtBeginSubsection{}

\setbeamertemplate{section in toc}[sections numbered]
\setbeamertemplate{subsection in toc}[subsections numbered]

%\insertsectionnumber

\setbeamertemplate{itemize items}[circle]

\usepackage{amssymb,amsmath,amsfonts,eurosym,geometry,graphicx,caption,color,setspace,
	comment,footmisc,caption,pdflscape,array}
\usepackage{booktabs}   % for nice tables
\usepackage{multirow}
%\usepackage[round]{natbib}
\setbeamertemplate{caption}[numbered]
\usepackage[export]{adjustbox}

\usepackage[skip=1pt]{caption}
%\usepackage[capposition=top]{floatrow}

%\usepackage[caption = false]{subfig}
%\usepackage{floatrow}
\usepackage[capposition=bottom]{floatrow}

\usepackage{graphicx}
\usepackage{tabularx}
%\usepackage{threeparttable}
\usepackage{float}
\usepackage{mwe}
%\usepackage{subfig}
%\usepackage{polyglossia}
\usepackage{subcaption}
\setlength{\abovecaptionskip}{2pt}
%\usepackage[tight,TABTOPCAP]{subfigure}
\usepackage[round]{natbib}

\usepackage{multicol, latexsym, amsmath, amssymb}

\usepackage[normalem]{ulem}
\useunder{\uline}{\ul}{}
\usepackage{booktabs,caption}
\usepackage[flushleft]{threeparttable}

\usepackage{colortbl}
\usepackage{tabu}
\usepackage{makecell}
\usepackage{xcolor}


\usepackage{graphics}

\usepackage{longtable}

\usepackage{float}

\usepackage{amsbsy} %boldsymbol

%%in case of outdated TEX Live
\usepackage{lmodern}

\usepackage{appendixnumberbeamer}

%\graphicspath{{figures/}{../figures/}{D:/Presentations\figures/}}
\usepackage[normalem]{ulem}

\usepackage{siunitx}
\sisetup{
	round-mode          = places, % Rounds numbers
	round-precision     = 4, % to 2 places
}


\usepackage{graphicx} % Allows including images
\usepackage{booktabs} % Allows the use of \toprule, \midrule and \bottomrule in tables

\usepackage{arydshln} %can use hdashline
\ifLuaTeX
  \usepackage{selnolig}  % disable illegal ligatures
\fi
\IfFileExists{bookmark.sty}{\usepackage{bookmark}}{\usepackage{hyperref}}
\IfFileExists{xurl.sty}{\usepackage{xurl}}{} % add URL line breaks if available
\urlstyle{same} % disable monospaced font for URLs
\hypersetup{
  pdftitle={Essays on Measuring Credit and Property Prices Gaps},
  pdfauthor={Nam Nguyen},
  hidelinks,
  pdfcreator={LaTeX via pandoc}}

\title{Essays on Measuring Credit and Property Prices Gaps}
\author{Nam Nguyen}
\date{August 11, 2022}
\institute{UWM}

\begin{document}
\frame{\titlepage}

\hypertarget{chapter-1-credit-and-house-prices-cycles}{%
\section{Chapter 1: Credit and House Prices
Cycles}\label{chapter-1-credit-and-house-prices-cycles}}

\begin{frame}{Introduction}
\protect\hypertarget{introduction}{}
\begin{block}{Motivation}
\protect\hypertarget{motivation}{}
\begin{itemize}
\item
  The study of housing prices and excessive credit has become more
  important in understanding financial market stability
\item
  We also observed increasing use of monetary policies, significant
  growth in macro balance sheet size, including real estate values and
  total credit lending to household
\item
  We study the dynamic relationship between housing prices and household
  credit in this paper
\end{itemize}
\end{block}
\end{frame}

\begin{frame}{Contribution}
\protect\hypertarget{contribution}{}
\begin{enumerate}
\tightlist
\item
  Relationship between housing prices and household credit
\end{enumerate}

\begin{itemize}
\item
  Apply Unobserved Component Model (Clark 1987) to extract information
  about trends and cycles

  \(\Rightarrow\) Jointly examine the two variables and their
  interaction both in the long-run and short-run
\item
  Specify cycles to be VAR process (cross-cycle) rather than univariate
  AR process

  \(\Rightarrow\) Test if past movement of one cycle has predictive
  power over another cycle
\end{itemize}
\end{frame}

\begin{frame}{Contribution}
\protect\hypertarget{contribution-1}{}
\begin{enumerate}
\setcounter{enumi}{1}
\tightlist
\item
  Technical contribution to the optimization process:
\end{enumerate}

\begin{itemize}
\tightlist
\item
  Novel numerical optimization / parameters constraint method to ensure
  the cyclical components are in feasible stationary region
\end{itemize}

\begin{enumerate}
\setcounter{enumi}{2}
\tightlist
\item
  Overcome ``curse of dimensionality'' using Bayesian method:
\end{enumerate}

\begin{itemize}
\tightlist
\item
  Common problem in estimating complex unobserved component state space
  model
\item
  We use random walk Metropolis-Hasting method to estimate posterior
  distribution of parameters of interest
\end{itemize}
\end{frame}

\begin{frame}{Literature Review}
\protect\hypertarget{literature-review}{}
\begin{enumerate}
\tightlist
\item
  Dynamics of credit changes:
\end{enumerate}

\begin{itemize}
\tightlist
\item
  Kiyotaki \& Moore (1997), Myerson (2012), Guerrieri \& Uhlig (2016),
  Boissay et al (2016).
\end{itemize}

\begin{enumerate}
\setcounter{enumi}{1}
\tightlist
\item
  Dynamics of house prices changes:
\end{enumerate}

\begin{itemize}
\tightlist
\item
  Hong \& Stein (1999), Glaeser et al (2008) (2017), Kishor, Kumari, \&
  Song (2015)
\end{itemize}
\end{frame}

\begin{frame}{Literature Review}
\protect\hypertarget{literature-review-1}{}
\begin{enumerate}
\setcounter{enumi}{2}
\tightlist
\item
  House price cycles generates credit cycles:
\end{enumerate}

\begin{itemize}
\tightlist
\item
  Bernanke \& Gertler (1989), Bernanke et al (1999); Kiyotaki \& Moore
  (1997) ``
\item
  Empirical Evidence: Fitzpatrick and McQuinn (2007), Berlinghieri
  (2010), Gimeno and Martinez-Carrascal (2010), Anundsen and Jansen
  (2013), for evidence from Ireland, USA, Spain and Norway, respectively
\end{itemize}

\begin{enumerate}
\setcounter{enumi}{3}
\tightlist
\item
  Credit cycles genereates house price cycles:
\end{enumerate}

\begin{itemize}
\tightlist
\item
  Agnello \& Schuknecht (2011), Kermani (2012), Justiniano et al (2019),
  Schularick et al (2012) (2016)
\end{itemize}

\(\Rightarrow\) However, the debate on which cycle causes changes on the
other is still open
\end{frame}

\begin{frame}{Data}
\protect\hypertarget{data}{}
Bank of International Settlement (BIS)

\begin{itemize}
\tightlist
\item
  Household Credit to GDP: Total Credit to non-financial sector
  (household)
\item
  House Price Index: Residential property prices: selected series (real
  value). Index = 100 at full sample average for each country \bigskip
\item
  2 countries: US \& UK
\item
  Time frame: 1990:Q1 - 2021:Q3
\end{itemize}
\end{frame}

\begin{frame}{Model}
\protect\hypertarget{model}{}
\begin{block}{Unobserved Component Model}
\protect\hypertarget{unobserved-component-model}{}
\begin{align}
    100*ln \frac{Credit}{GDP} &= y_t = \tau_{yt} + c_{yt}
    \\
    100*ln HPI &= h_t = \tau_{ht} + c_{ht}
\end{align}

\begin{itemize}
\tightlist
\item
  Trends: \(\tau_{yt}\) \& \(\tau_{ht}\)
\end{itemize}

\begin{align*}
        \tau_{yt} = &\mu_{yt-1} + \tau_{yt-1} +  \eta_{yt}, &\eta_{yt} \sim iidN(0,\sigma^2_{\eta y})
\\
&\mu_{yt} = \mu_{yt-1} + \eta_{\mu yt}, &\eta_{\mu yt} \sim iidN(0,0.01)
\\
\tau_{ht} = &\mu_{ht-1} + \tau_{ht-1} + \eta_{ht}, &\eta_{ht} \sim iidN(0,\sigma^2_{\eta h})
\\
&\mu_{ht} = \mu_{ht-1} + \eta_{\mu ht}, &\eta_{\mu ht} \sim iidN(0,0.01)  
\end{align*}
\end{block}
\end{frame}

\begin{frame}{Model}
\protect\hypertarget{model-1}{}
\begin{block}{Unobserved Component Model}
\protect\hypertarget{unobserved-component-model-1}{}
\begin{itemize}
\tightlist
\item
  Cycles: \(c_{yt}\) \& \(c_{ht}\)
\end{itemize}

\begin{align}
    c_{yt} = \phi^1_{y}c_{yt-1}  
    + \phi^2_{y}c_{yt-2}  
    + \phi^{x1}_{y}c_{ht-1} 
    + &\phi^{x2}_{y}c_{ht-1} 
    + \varepsilon_{yt}\\
    &\varepsilon_{yt} \sim iidN(0,\sigma^2_{\varepsilon y})        
    \\
    c_{ht} = \phi^1_{h}c_{ht-1}  
    + \phi^2_{h}c_{ht-2}
    + \phi^{x1}_{h}c_{yt-1}  
    + &\phi^{x2}_{h}c_{yt-1}  
    + \varepsilon_{ht}\\
    &\varepsilon_{ht} \sim iidN(0,\sigma^2_{\varepsilon h})
\end{align}
\end{block}
\end{frame}

\begin{frame}{Model}
\protect\hypertarget{model-2}{}
\begin{block}{Covariance Matrix}
\protect\hypertarget{covariance-matrix}{}
\begin{align}
    Q = 
    \begin{bmatrix}
    \sigma^2_{\eta y} & 0  &0 & \sigma_{\eta y \eta h}  & 0 & 0 & 0 & 0 \\
    0 & \sigma^2_{\varepsilon y}  & 0 & 0 & \sigma_{\varepsilon y \varepsilon h}  & 0 & 0 & 0 \\
    0 & 0 & 0 & 0 & 0 & 0 & 0 & 0 \\
    \sigma_{\eta y \eta h}  & 0 & 0 & \sigma^2_{\eta h} & 0 & 0 & 0 & 0 \\
    0 & \sigma_{\varepsilon y \varepsilon h}  & 0 & 0 & \sigma^2_{\varepsilon h} & 0  & 0 & 0 \\
    0 & 0 & 0 & 0 & 0 & 0 & 0 & 0 \\
    0 & 0 & 0 & 0 & 0 & 0 & 0.01 & 0 \\
    0 & 0 & 0 & 0 & 0 & 0 & 0 & 0.01
    \end{bmatrix}
\end{align}
\end{block}
\end{frame}

\begin{frame}{Model}
\protect\hypertarget{model-3}{}
\begin{block}{Optimization process}
\protect\hypertarget{optimization-process}{}
\begin{itemize}
\tightlist
\item
  Kalman filter with adjusted Likelihood function:
\end{itemize}

\begin{align*}
l(\theta) = -0.5\sum_{t=1}^{T}ln\lbrack(2\pi)^2|f_{t|t-1}|\rbrack
                -0.5\sum_{t=1}^{T}\eta'_{t|t-1}f^{-1}_{t|t-1}\eta_{t|t-1}
                \\
                - w1\sum_{t=1}^{T}(c_{yt}^2) - w2\sum_{t=1}^{T}(c_{ht}^2)
\end{align*}
\end{block}
\end{frame}

\hypertarget{empirical-results}{%
\section{Empirical Results}\label{empirical-results}}

\begin{frame}{VAR(2) - 1 Cross-lag Model Estimate - UK and US}
\protect\hypertarget{var2---1-cross-lag-model-estimate---uk-and-us}{}
\resizebox{\linewidth}{!}{
\begin{tabular}[t]{>{}l>{}l>{}r>{}l>{}r>{}l}
\toprule
\multicolumn{2}{c}{ } & \multicolumn{2}{c}{UK VAR2 1-cross lag} & \multicolumn{2}{c}{US VAR2 1-cross lag} \\
\cmidrule(l{3pt}r{3pt}){3-4} \cmidrule(l{3pt}r{3pt}){5-6}
Description & Para. & Median & {}[10$\%$, 90$\%$] & Median & {}[10$\%$, 90$\%$]\\
\midrule
\cellcolor{gray!6}{Credit to household 1st AR parameter} & \cellcolor{gray!6}{$\phi^1_{y}$} & \cellcolor{gray!6}{1.4238} & \cellcolor{gray!6}{{}[1.3585, 1.4892]} & \cellcolor{gray!6}{1.2074} & \cellcolor{gray!6}{{}[1.1374, 1.2785]}\\
Credit to household 2nd AR parameter & $\phi^2_{y}$ & -0.4698 & {}[-0.5305, -0.4090] & -0.2483 & {}[-0.3152, -0.1825]\\
\textbf{\cellcolor{gray!6}{Credit to household 1st cross cycle AR parameter}} & \textbf{\cellcolor{gray!6}{$\phi^{x1}_{y}$}} & \textbf{\cellcolor{gray!6}{0.0238}} & \textbf{\cellcolor{gray!6}{{}[0.0154, 0.0319]}} & \textbf{\cellcolor{gray!6}{0.0318}} & \textbf{\cellcolor{gray!6}{{}[0.0228, 0.0407]}}\\
Credit to household 2nd cross cycle AR parameter & $\phi^{x2}_{y}$ &  &  &  & \\
\addlinespace
\cellcolor{gray!6}{Housing Price Index 1st AR parameter} & \cellcolor{gray!6}{$\phi^1_{h}$} & \cellcolor{gray!6}{1.3173} & \cellcolor{gray!6}{{}[1.2647, 1.3701]} & \cellcolor{gray!6}{1.8038} & \cellcolor{gray!6}{{}[1.7700, 1.8363]}\\
Housing Price Index 2nd AR parameter & $\phi^2_{h}$ & -0.3315 & {}[-0.3885, -0.2746] & -0.8261 & {}[-0.8605, -0.7903]\\
\textbf{\cellcolor{gray!6}{Housing Price Index 1st cross cycle AR parameter}} & \textbf{\cellcolor{gray!6}{$\phi^{x1}_{h}$}} & \textbf{\cellcolor{gray!6}{-0.0173}} & \textbf{\cellcolor{gray!6}{{}[-0.0464, 0.0062]}} & \textbf{\cellcolor{gray!6}{0.0104}} & \textbf{\cellcolor{gray!6}{{}[0.0007, 0.0204]}}\\
Housing Price Index 2nd cross cycle AR parameter & $\phi^{x2}_{h}$ &  &  &  & \\
\addlinespace
\cellcolor{gray!6}{S.D. of permanent shocks to Credit to household} & \cellcolor{gray!6}{$\sigma_{ny}$} & \cellcolor{gray!6}{0.2714} & \cellcolor{gray!6}{{}[0.2150, 0.3155]} & \cellcolor{gray!6}{0.2954} & \cellcolor{gray!6}{{}[0.2312, 0.3414]}\\
S.D. of transitory shocks to Credit to household & $\sigma_{ey}$ & 0.8021 & {}[0.7699, 0.8376] & 0.8631 & {}[0.8287, 0.9012]\\
\cellcolor{gray!6}{S.D. of permanent shocks to Housing Price Index} & \cellcolor{gray!6}{$\sigma_{nh}$} & \cellcolor{gray!6}{0.0789} & \cellcolor{gray!6}{{}[0.0742, 0.0845]} & \cellcolor{gray!6}{0.1390} & \cellcolor{gray!6}{{}[0.1222, 0.1618]}\\
S.D. of transitory shocks to Housing Price Index & $\sigma_{eh}$ & 1.2242 & {}[1.1886, 1.2613] & 0.8988 & {}[0.8641, 0.9355]\\
\addlinespace
\cellcolor{gray!6}{Correlation: Permanent credit to household$\slash$Permanent HPI} & \cellcolor{gray!6}{$\rho_{nynh}$} & \cellcolor{gray!6}{0.0189} & \cellcolor{gray!6}{{}[-0.3049, 0.3393]} & \cellcolor{gray!6}{0.0082} & \cellcolor{gray!6}{{}[-0.3117, 0.3226]}\\
Correlation: Transitory credit to household$\slash$Transitory HPI & $\rho_{eyeh}$ & 0.2536 & {}[0.1713, 0.3337] & 0.1537 & {}[0.0399, 0.2619]\\
\cellcolor{gray!6}{Log-likelihood value} & \cellcolor{gray!6}{$llv$} & \cellcolor{gray!6}{578.6200} & \cellcolor{gray!6}{{}[576.1600, 582.1500]} & \cellcolor{gray!6}{204.9400} & \cellcolor{gray!6}{{}[202.4200, 208.4500]}\\
\bottomrule
\multicolumn{6}{l}{\rule{0pt}{1em}\textit{Note: }}\\
\multicolumn{6}{l}{\rule{0pt}{1em}UK - US Bayesian method random walk  Metropolis-Hasting posterior distribution estimates}\\
\end{tabular}}
\end{frame}

\begin{frame}{VAR(2) - 1 Cross-lag Model Estimate Summary}
\protect\hypertarget{var2---1-cross-lag-model-estimate-summary}{}
\begin{itemize}
\item
  The sum of AR parameters of the cyclical components in all three
  models is smaller, although close to one
\item
  The standard deviation of the shocks in the cycles \(\sigma_{ei}\) is
  much higher than the standard deviation of the shocks to the trend
  \(\sigma_{ni}\) of both credit and housing prices
\item
  Variations in the housing price cyclical components \(\sigma_{eh}\) of
  the UK are bigger than in the US
\item
  The correlation of the shocks to the cyclical components among the two
  variables \(\rho_{eyeh}\) suggests that cyclical variation among
  housing price and household credit is strongly positively correlated
\end{itemize}
\end{frame}

\begin{frame}{Cross-country Comparison of Causal Coefficients}
\protect\hypertarget{cross-country-comparison-of-causal-coefficients}{}
\begingroup\fontsize{7}{9}\selectfont

\begin{tabular}[t]{lrlrl}
\toprule
\multicolumn{1}{c}{ } & \multicolumn{2}{c}{$\phi^{x1}_y$ HPI on Credit} & \multicolumn{2}{c}{$\phi^{x1}_h$ Credit on HPI} \\
\cmidrule(l{3pt}r{3pt}){2-3} \cmidrule(l{3pt}r{3pt}){4-5}
Country & Median & {}[10$\%$, 90$\%$] & Median & {}[10$\%$, 90$\%$]\\
\midrule
\cellcolor{gray!6}{Australia} & \cellcolor{gray!6}{0.0157} & \cellcolor{gray!6}{{}[-0.0093, 0.0412]} & \cellcolor{gray!6}{0.0521} & \cellcolor{gray!6}{{}[0.0014, 0.1060]}\\
Belgium & 0.0279 & {}[0.0013, 0.0559] & -0.0656 & {}[-0.0980, -0.0339]\\
\cellcolor{gray!6}{Canada} & \cellcolor{gray!6}{0.0191} & \cellcolor{gray!6}{{}[0.0032, 0.0332]} & \cellcolor{gray!6}{-0.0152} & \cellcolor{gray!6}{{}[-0.0343, 0.0025]}\\
Finland & 0.0080 & {}[0.0017, 0.0156] & 0.0085 & {}[0.0021, 0.0156]\\
\addlinespace
\cellcolor{gray!6}{France} & \cellcolor{gray!6}{0.0298} & \cellcolor{gray!6}{{}[0.0185, 0.0411]} & \cellcolor{gray!6}{-0.0643} & \cellcolor{gray!6}{{}[-0.1098, -0.0241]}\\
Germany & 0.0728 & {}[0.0500, 0.0917] & -0.0061 & {}[-0.0282, 0.0052]\\
\cellcolor{gray!6}{Hong Kong} & \cellcolor{gray!6}{-0.0031} & \cellcolor{gray!6}{{}[-0.0079, 0.0019]} & \cellcolor{gray!6}{-0.0629} & \cellcolor{gray!6}{{}[-0.0836, -0.0453]}\\
Italy & 0.1001 & {}[0.0895, 0.1063] & -0.0027 & {}[-0.0072, 0.0014]\\
\addlinespace
\cellcolor{gray!6}{Japan} & \cellcolor{gray!6}{-0.0088} & \cellcolor{gray!6}{{}[-0.0326, 0.0174]} & \cellcolor{gray!6}{0.1659} & \cellcolor{gray!6}{{}[0.1202, 0.2173]}\\
Netherlands & 0.0058 & {}[-0.0039, 0.0166] & -0.0043 & {}[-0.0156, 0.0070]\\
\cellcolor{gray!6}{New Zealand} & \cellcolor{gray!6}{0.0078} & \cellcolor{gray!6}{{}[-0.0035, 0.0199]} & \cellcolor{gray!6}{-0.0139} & \cellcolor{gray!6}{{}[-0.0249, -0.0036]}\\
Norway & 0.0109 & {}[0.0097, 0.0116] & 0.0059 & {}[0.0047, 0.0066]\\
\addlinespace
\cellcolor{gray!6}{South Korea} & \cellcolor{gray!6}{0.0106} & \cellcolor{gray!6}{{}[-0.0033, 0.0308]} & \cellcolor{gray!6}{0.0027} & \cellcolor{gray!6}{{}[-0.0251, 0.0369]}\\
Spain & 0.0144 & {}[0.0003, 0.0331] & 0.0051 & {}[-0.0023, 0.0146]\\
\cellcolor{gray!6}{Sweden} & \cellcolor{gray!6}{0.0159} & \cellcolor{gray!6}{{}[0.0071, 0.0252]} & \cellcolor{gray!6}{0.0400} & \cellcolor{gray!6}{{}[0.0218, 0.0617]}\\
United Kingdom & 0.0238 & {}[0.0154, 0.0319] & -0.0173 & {}[-0.0464, 0.0062]\\
\addlinespace
\cellcolor{gray!6}{United States} & \cellcolor{gray!6}{0.0318} & \cellcolor{gray!6}{{}[0.0228, 0.0407]} & \cellcolor{gray!6}{0.0104} & \cellcolor{gray!6}{{}[0.0007, 0.0204]}\\
\bottomrule
\end{tabular}
\endgroup{}
\end{frame}

\begin{frame}{Cross-country Comparison of Causal Coefficients Summary}
\protect\hypertarget{cross-country-comparison-of-causal-coefficients-summary}{}
\begin{itemize}
\item
  In 11 out of 17 countries, the HPI on Credit causal coefficient
  \(\phi^{x1}_y\) are positive and significant. All 11 countries are in
  North America and Europe.
\item
  Only 6 countries have positive and significant Credit on HPI causal
  coefficient \(\phi^{x1}_y\). Three of which have smaller magnitudes
  than their \(\phi^{x1}_y\) counterpart
\end{itemize}

\(\rightarrow\) Overall, we found evidence that past transitory shocks
to house price credit will cause a positive deviation in future
transitory household credit. However, the effect in the opposite
direction is much smaller and sometimes insignificant
\end{frame}

\begin{frame}{Unobserved Component Graphs: United States}
\protect\hypertarget{unobserved-component-graphs-united-states}{}
\includegraphics{../../HPCredit/Regression/Bayesian_UC_VAR2_drift_Crosscycle1lag/OutputData/graphs/HP_Credit_4graphs_US.pdf}
\end{frame}

\begin{frame}{Unobserved Component Graphs: United Kingdom}
\protect\hypertarget{unobserved-component-graphs-united-kingdom}{}
\includegraphics{../../HPCredit/Regression/Bayesian_UC_VAR2_drift_Crosscycle1lag/OutputData/graphs/HP_Credit_4graphs_UK.pdf}
\end{frame}

\begin{frame}{Conclusion}
\protect\hypertarget{conclusion}{}
\begin{itemize}
\tightlist
\item
  Extracting temporary and permanent components information gave
  insights on the dynamics of the two series housing and credit in both
  short-run and long-run
\item
  Evidence showing that past movement of a cycle (HPI) has predictive
  power over the other cycle (credit)
\end{itemize}
\end{frame}

\hypertarget{chapter-2-measuring-credit-gaps}{%
\section{Chapter 2: Measuring Credit
Gaps}\label{chapter-2-measuring-credit-gaps}}

\begin{frame}{Introduction}
\protect\hypertarget{introduction-1}{}
\begin{block}{Motivation}
\protect\hypertarget{motivation-1}{}
\begin{itemize}
\item
  No unanimity on how to measure excessive credit. Bank for
  International Settlements uses HP filter to create a credit gap
  measurement that performs well in predicting future financial crises.
  However, there are other competing gap measurements.
\item
  Nelson (2008) that the deviation of a non-stationary variable from its
  long-run trend should predict future changes of opposite sign in the
  variable. We utilize this idea and forecast combination method to
  propose a synthesized credit gap measurement.
\end{itemize}
\end{block}
\end{frame}

\begin{frame}{Contribution}
\protect\hypertarget{contribution-2}{}
Since different trend-cycle decomposition methods of credit-to-GDP ratio
provide us different credit gap measures, we handle the model
uncertainty by assigning weights on these different credit gap measures
based on its relative out-of-sample predictive power based on Bates and
Granger (1969) forecast combination method.

\begin{itemize}
\tightlist
\item
  Our proposed credit gap measure dominates the alternate credit gaps in
  terms of its relative out-of-sample predictive power.
\end{itemize}
\end{frame}

\begin{frame}{Methodology}
\protect\hypertarget{methodology}{}
\begin{block}{Data}
\protect\hypertarget{data-1}{}
The measure of credit is total credit to the private non-financial
sector, as published in the BIS database, capturing total borrowing from
all domestic and foreign sources.

\begin{itemize}
\tightlist
\item
  Quarterly data from 1983:Q1-2020:Q2
\end{itemize}
\end{block}
\end{frame}

\begin{frame}{Model}
\protect\hypertarget{model-4}{}
All these trend-cycle decomposition methods are based on the premise
that a non-stationary series is the sum of a trend and a stationary
cyclical component:

\begin{equation}
y_{t}=\tau _{t}+c_{t}
\end{equation}
\end{frame}

\begin{frame}{Trend-cycle decomposition models}
\protect\hypertarget{trend-cycle-decomposition-models}{}
\begin{block}{HP filter}
\protect\hypertarget{hp-filter}{}
\begin{equation}
min_{\tau}(\sum^T_{t=1}(y_t-\tau_t)^2+\lambda\sum^{T-1}_{t=2}[(\tau_{t-1}-\tau_t)-(\tau_t-\tau_{t-1})]^2)
\end{equation}

\begin{itemize}
\tightlist
\item
  \(\lambda\) will be set at 1600, 3000, 400000 (BIS Basel Gap) in our
  models
\end{itemize}
\end{block}

\begin{block}{Unobserved-Component model: Clark(1987)}
\protect\hypertarget{unobserved-component-model-clark1987}{}
\begin{equation}
\tau _{t}=\tau _{t-1}+\eta _{t},\eta _{t}\symbol{126}iid(0,\sigma _{\eta
}^{2})
\end{equation} \begin{equation*}
c_{t}=\Phi (L)c_{t}+u_{t},u_{t}\symbol{126}iid(0,\sigma _{u}^{2})
\end{equation*}
\end{block}
\end{frame}

\begin{frame}{Trend-cycle decomposition models}
\protect\hypertarget{trend-cycle-decomposition-models-1}{}
\begin{block}{Beveridge-Nelson}
\protect\hypertarget{beveridge-nelson}{}
\begin{equation}
y_{t}=y_{0}+\mu t+\Psi (1)\sum_{k=1}^{t}u_{t}+\overset{\backsim }{u_{t}}-%
\overset{\backsim }{u_{0}}
\end{equation}
\end{block}

\begin{block}{Hamilton filter (2018)}
\protect\hypertarget{hamilton-filter-2018}{}
\begin{equation}
y_{t+h}=\alpha +\beta _{1}y_{t-1}+\beta _{2}y_{t-2}+....+\beta
_{p}y_{t-p}+v_{t+h}
\end{equation}
\end{block}
\end{frame}

\begin{frame}{Forecasting model:}
\protect\hypertarget{forecasting-model}{}
\begin{equation}
\Delta y_{t}=\alpha +\beta (L)\Delta y_{t-1}+\gamma (L)GAP_{t-1}+v_{t}
\end{equation}

\begin{block}{Baseline Model AR(1):}
\protect\hypertarget{baseline-model-ar1}{}
\begin{equation}
\Delta y_{t}=\alpha +\beta (L)\Delta y_{t-1}+v_{t}
\end{equation}
\end{block}

\begin{block}{Forecast combination}
\protect\hypertarget{forecast-combination}{}
\begin{equation}
w_{m}=\frac{\widehat{\overline{\sigma }}_{m}^{2}}{\widehat{\overline{\sigma }%
}_{1}^{2}+\widehat{\overline{\sigma }}_{2}^{2}+....\widehat{\overline{\sigma 
}}_{M}^{2}}
\end{equation}

\begin{itemize}
\tightlist
\item
  where \(\widehat{\overline{\sigma }}_{m}^{2}\)~is inverted
  out-of-sample forecast error variance of forecast M based on the
  cyclical component M.
\end{itemize}
\end{block}
\end{frame}

\begin{frame}{Empirical Results:}
\protect\hypertarget{empirical-results-1}{}
\begin{block}{Forecasting Performance of Credit Gap Models (U.S.)}
\protect\hypertarget{forecasting-performance-of-credit-gap-models-u.s.}{}
\resizebox{\linewidth}{!}{
\begin{threeparttable}
\begin{tabular}{lllllllllll}
Horizon & HP & RU & BIS & Hamilton & \ Linear\  & Quadratic & BN & UC & 
Average & Bates-Granger \\ \hline \hline
\  \  \  \  \  \ 1 & 0.993 & 0.987 & 1.012 & 0.994 & 1.028 & 1.005 & 1.010 & 0.985
& 0.962 & \textbf{0.959} \\ 
\  \  \  \  \  \ 2 & 0.974 & 0.963 & 1.016 & 0.980 & 1.058 & 1.014 & 0.975 & 0.961
& 0.924 & \textbf{0.917} \\ 
\  \  \  \  \  \ 3 & 0.966 & 0.953 & 1.023 & 1.011 & 1.055 & 1.036 & 0.965 & 0.937
& 0.906 & \textbf{0.896} \\ 
\  \  \  \  \  \ 4 & 0.982 & 0.966 & 1.022 & 1.029 & 1.055 & 1.045 & 1.033 & 0.910
& 0.922 & \textbf{0.910} \\ 
\  \  \ 1 - 4 & 0.964 & 0.945 & 1.030 & 1.005 & 1.081 & 1.041 & 0.978 & 0.913
& 0.882 & \textbf{0.872} \\ \hline \hline
&  &  &  &  &  &  &  &  &  & 
\end{tabular}
\begin{tablenotes}
            \footnotesize
            \item {The table shows the ratio of RMSEPs of different models in comparison
to the benchmark AR(1) model. The first set of forecasts is for
1994:Q1-1994:Q4; the final set is for 2019:Q3-2020:Q2. Q=1-4 denotes
averages over next 4-quarters. HP is Hodrick-Prescott, RU is Ravn-Uhlig, BIS
is based on Borio and Lowe (2002), BN is Beveridge-Nelson, UC is Unobserved
Component Model.}
        \end{tablenotes}
\end{threeparttable}}
\end{block}
\end{frame}

\begin{frame}{Empirical Results:}
\protect\hypertarget{empirical-results-2}{}
\begin{block}{Forecasting Performance of Credit Gap Models (U.K.)}
\protect\hypertarget{forecasting-performance-of-credit-gap-models-u.k.}{}
\resizebox{\linewidth}{!}{
\begin{threeparttable}
\begin{tabular}{lllllllllll}
Horizon & HP & RU & BIS & Hamilton & \ Linear\  & Quadratic & BN & UC & 
Average & Bates-Granger \\ \hline \hline
\  \  \  \  \  \ 1 & 1.001 & 0.990 & 1.001 & 0.992 & 1.010 & 0.979 & 1.028 & 1.009
& \textbf{0.977} & 0.979 \\ 
\  \  \  \  \  \ 2 & 0.979 & 0.970 & 1.007 & 0.969 & 1.016 & 0.962 & 1.028 & 0.999
& 0.962 & \textbf{0.957} \\ 
\  \  \  \  \  \ 3 & 0.979 & 0.971 & 1.018 & 0.969 & 1.055 & 0.966 & 1.009 & 0.989
& 0.959 & \textbf{0.955} \\ 
\  \  \  \  \  \ 4 & 0.990 & 0.987 & 1.028 & 1.005 & 1.055 & 0.981 & 1.019 & 0.981
& 0.972 & \textbf{0.967} \\ 
\  \  \ 1 - 4 & 0.972 & 0.952 & 1.034 & 0.960 & 1.081 & 0.929 & 1.054 & 0.985
& 0.918 & \textbf{0.910} \\ \hline \hline
&  &  &  &  &  &  &  &  &  & \\
\end{tabular}
\begin{tablenotes}
            \footnotesize
            \item {The table shows the ratio of RMSEPs of different models in comparison
to the benchmark AR(1) model. The first set of forecasts is for
1994:Q1-1994:Q4; the final set is for 2019:Q3-2020:Q2. Q=1-4 denotes
averages over next 4-quarters. HP is Hodrick-Prescott, RU is Ravn-Uhlig, BIS
is based on Borio and Lowe (2002), BN is Beveridge-Nelson, UC is Unobserved
Component Model.}
        \end{tablenotes}
\end{threeparttable}}
\end{block}
\end{frame}

\begin{frame}{Credit Gap Comparison (U.S.)}
\protect\hypertarget{credit-gap-comparison-u.s.}{}
\includegraphics[width=0.85\linewidth]{../../HPI-Credit-Trasitory-Forecast/plots/Credit-Gap-Comparison-US}
\end{frame}

\begin{frame}{Credit Gap Comparison (U.K.)}
\protect\hypertarget{credit-gap-comparison-u.k.}{}
\includegraphics[width=0.85\linewidth]{../../HPI-Credit-Trasitory-Forecast/plots/Credit-Gap-Comparison-UK}
\end{frame}

\begin{frame}{Conclusion}
\protect\hypertarget{conclusion-1}{}
Our results show that this method of combining credit gaps yield us a
credit gap measure that dominates credit gaps from different trend-cycle
decomposition methods in terms of superior out-of-sample forecasting of
changes in credit-to-GDP ratio.
\end{frame}

\hypertarget{chapter-3-identifying-unsustainable-credit-gaps}{%
\section{Chapter 3: Identifying Unsustainable Credit
Gaps}\label{chapter-3-identifying-unsustainable-credit-gaps}}

\begin{frame}{Motivation}
\protect\hypertarget{motivation-2}{}
\begin{itemize}
\tightlist
\item
  To overcome model uncertainty in using credit gap as an early warning
  indicator (EWI) of systemic financial crises, we propose using model
  averaging of different credit gap measurements. The method is based on
  Bayesian Model Average - Raftery (1995)
\end{itemize}
\end{frame}

\begin{frame}{Motivation from Literature}
\protect\hypertarget{motivation-from-literature}{}
\begin{itemize}
\tightlist
\item
  Area under the curve of operating characteristic (AUROC or AUC) has
  been widely used as a criterion to determine the performance of a EWI.
  But it has recently received some criticism.
\item
  Borio and Drehmann (2009) and Beltran et al (2021) proposed a policy
  loss function constraining the relevance of the curve measurement to
  just a portion where Type II error rate is less than 1/3 or at least
  2/3 of the crises are predicted.\\
\item
  Detken (2014) proposed using partial standardized area under the curve
  (psAUC) as an alternative measurement of the performance of an EWI.
\end{itemize}
\end{frame}

\begin{frame}{Contribution}
\protect\hypertarget{contribution-3}{}
\begin{itemize}
\tightlist
\item
  Compare different credit gap measurements' performance as EWIs using a
  new criterion - partial standarized AUC (psAUC) contraining Type II
  error \textless{} 1/3.
\item
  Overcome model uncertainty by implementing model averaging. We
  incoporated psAUC values in the model selection and weighting process,
  instead of AUC values.
\item
  For ease of policy implication, we propose a single credit gap
  measurement from weighted averaging other popularly studied credit gap
  measurements. The gap has superior performance in model fit and
  out-of-sample prediction.
\end{itemize}
\end{frame}

\begin{frame}{standardize psAUROC - Detken (2014)}
\protect\hypertarget{standardize-psauroc---detken-2014}{}
\begin{center}\includegraphics[width=0.7\linewidth]{../../3rdpaper/metadata/pAUC} \end{center}

\begin{align}
psAUROC = \frac{1}{2}\left[ 1+ \frac{pAUROC - min}{max - min}\right]
\end{align}
\end{frame}

\begin{frame}{Data}
\protect\hypertarget{data-2}{}
Sample data periods:

\begin{itemize}
\tightlist
\item
  1970:Q4 - 2017:Q4 quarterly data across 43 countries.

  \begin{itemize}
  \tightlist
  \item
    We omit periods for countries with shorter credit measurements.
  \end{itemize}
\end{itemize}

Systemic crisis data:

\begin{itemize}
\tightlist
\item
  European Systemic Risk Board crisis data set (Lo Duca et al.~2017)
\item
  Laeven and Valencia (2018)
\end{itemize}

Credit/GDP ratio data:

\begin{itemize}
\tightlist
\item
  Bank of International Settlement (BIS)

  \begin{itemize}
  \tightlist
  \item
    Latest credit data is available until 2021:Q3
  \end{itemize}
\end{itemize}
\end{frame}

\begin{frame}{Emprical Model}
\protect\hypertarget{emprical-model}{}
\begin{block}{Credit gap decompositions}
\protect\hypertarget{credit-gap-decompositions}{}
\begin{align}
    100*\frac{Credit}{GDP} &= y_t = \tau_{yt} + c_{yt}
\end{align}

\begin{itemize}
\tightlist
\item
  We created 90 candidate one-sided credit gap measurements based on the
  literature.

  \begin{itemize}
  \tightlist
  \item
    Once a country has more than 15 years of credit measurement
    available, we start storing its one-sided credit gap values onward.
  \end{itemize}
\end{itemize}
\end{block}
\end{frame}

\begin{frame}{Early Warning Indicator - Logistic regression:}
\protect\hypertarget{early-warning-indicator---logistic-regression}{}
\begin{align}
  pre.crisis_{ti} \sim credit.gap_{tij}
\end{align}

\begin{itemize}
\item
  \(i\) is country indicator. \(j\) is credit gap filter type
\item
  where \(pre.crisis_{it}=\) 1 or 0
\item
  The pre-crisis indicator is set to 1 when t is between 5-12 quarters
  before a systemic crisis.
\item
  We discard measurements between 1-4 quarters before a crisis, periods
  during a crisis and post-crisis periods identified in Lo Duca et
  al.~(2017) and Laeven and Valencia (2018).

  \begin{itemize}
  \tightlist
  \item
    The indicator is set to 0 at other periods.
  \item
    pre-crisis periods of imported crises identified in the dataset are
    also set to 0. However, we still discard measurements of periods
    during and post-crisis for imported crises.
  \end{itemize}
\end{itemize}
\end{frame}

\begin{frame}{Plot of Credit gap, Threshold and Crisis periods}
\protect\hypertarget{plot-of-credit-gap-threshold-and-crisis-periods}{}
\begin{center}\includegraphics[width=1\linewidth]{../../3rdPaper/Data/Output/Graphs/Weighted_credit_gap_US} \end{center}
\end{frame}

\begin{frame}{Variable selection}
\protect\hypertarget{variable-selection}{}
\begin{block}{Comparing performance of individual credit gaps}
\protect\hypertarget{comparing-performance-of-individual-credit-gaps}{}
Using partial area under the curve (psAUC) values
\end{block}

\begin{block}{Test for gaps combination performance}
\protect\hypertarget{test-for-gaps-combination-performance}{}
Using Markov Chain Monte Carlo Model Comparison (\(MC^3\)) developed by
Madigan and York (1995). The method assigns posterior probability for
different credit gaps being selected in most likely models/combinations.
Babecky (2014) used this \(MC^3\) method to identify potential variables
in EWI models.

\begin{align*}
Model_k :  pre.crisis_{ti} \sim \sum\nolimits_j \beta_j * credit.gap_{tij}
\end{align*}
\end{block}

\begin{block}{Variable selection}
\protect\hypertarget{variable-selection-1}{}
We selected 29 credit gap measurements based on these 2 criteria.
\end{block}
\end{frame}

\hypertarget{model-averaging}{%
\section{Model Averaging}\label{model-averaging}}

\begin{frame}{Bayesian Model Averging}
\protect\hypertarget{bayesian-model-averging}{}
The Bayesian Model Average method is formalized in Raftery (1995) to
account for model uncertainty.

\begin{block}{Model posterior probability}
\protect\hypertarget{model-posterior-probability}{}
Model k posterior probability (weight): \begin{align}
  P(M_k|D) = \frac{P(D|M_k)P(M_k)}{\sum\nolimits_{l=1}^K P(D|M_l)P(M_l)} 
  \approx \frac{exp(-\frac{1}{2}BIC_k)}{\sum\nolimits_{l=1}^K exp(-\frac{1}{2}BIC_l)}
\end{align}

\begin{itemize}
\item
  Where \(P(M_k)\) is model prior probability and can be ignored if all
  models are assumed equal prior weights.
\item
  \(P(D|M_k)\) is marginal likehood. And
  \(P(D|M_k) \propto exp(-\frac{1}{2}BIC_k)\)
\item
  In which
  \(BIC_k = 2log (Bayesfactor_{sk}) = \chi^2_{sk} - df_klog(n)\). s
  indicates the saturated model.
\end{itemize}
\end{block}
\end{frame}

\begin{frame}{Weighted credit gap creation}
\protect\hypertarget{weighted-credit-gap-creation}{}
\begin{block}{Motivation}
\protect\hypertarget{motivation-3}{}
GLM binomial estimation: \begin{align*}
\widehat{pre.crisis}_{ti} = \widehat{probability}_{ti} = \frac {1}{1+exp(-(a+\sum\nolimits_j \hat{\beta}_j c_{tij}))}
\end{align*}

\begin{itemize}
\tightlist
\item
  With \(\hat{\beta}_j\) =
  \(E[\beta_j|D, \beta_j\ne 0] = \sum\limits_{A_j} \hat{\beta}_j(k)p'(M_k|D)\)
\end{itemize}

\(\Rightarrow\) We propose a single weighted credit gap \(\hat{c}_{ti}\)
that satisfies: \begin{align*}
\frac {1}{1+exp(-(a+\hat{\beta} \hat{c}_{ti}))}= \frac {1}{1+exp(-(a+\sum\nolimits_j \hat{\beta}_j c_{tij}))} \\
\end{align*} OR \begin{align}
\sum\limits_j \hat{\beta}_j c_{tij} = \hat{\beta} \hat{c}_{ti}
\end{align}
\end{block}
\end{frame}

\begin{frame}{Weighted averaged credit gap - creation}
\protect\hypertarget{weighted-averaged-credit-gap---creation}{}
\begin{align*}
\sum\limits_j \hat{\beta}_j c_{tij} = \hat{\beta} \hat{c}_{ti}
\end{align*}

We then propose \(\hat{\beta} = \sum\nolimits_j \hat{\beta}_j\)

Therefore,

\begin{align}
\hat{c}_{ti} = \frac{\sum\nolimits_j (\hat{\beta}_j c_{tij})}{\sum\nolimits_j\hat{\beta}_j} = \sum\nolimits_j w_j c_{tij}
\end{align}

The weight of each candidate credit gap j is
\(w_j = \frac{\hat{\beta}_j}{\sum\nolimits_j\hat{\beta}_j}\)
\end{frame}

\begin{frame}{One-sided crisis weighted averaged credit gap}
\protect\hypertarget{one-sided-crisis-weighted-averaged-credit-gap}{}
\begin{itemize}
\item
  The weight of each candidate credit gap j is
  \(w_j = \frac{\hat{\beta}_j}{\sum\nolimits_j\hat{\beta}_j}\)
\item
  We save the weights \(w_j\) at every incremental period \(t\) of
  available data to create a one-sided weight vector \(w_{tj}\).
\end{itemize}

\(\Rightarrow\) To create one-sided crisis weighted averaged credit gap
for each country \(i\) (\(\hat{c}_{ti}\)), we compute: \begin{align}
\hat{c}_{ti,one-sided} = \sum\nolimits_{j} w_{tj} * c_{tij}
\end{align}
\end{frame}

\hypertarget{empirical-results-3}{%
\section{Empirical Results}\label{empirical-results-3}}

\begin{frame}{Weights stacked graph}
\protect\hypertarget{weights-stacked-graph}{}
Weights are restricted to be positive to ensure stability

\begin{center}\includegraphics[width=1\linewidth]{../../3rdPaper/Data/Output/Graphs/Weights_stack} \end{center}
\end{frame}

\begin{frame}{Weights series graph}
\protect\hypertarget{weights-series-graph}{}
\begin{center}\includegraphics[width=1\linewidth]{../../3rdPaper/Data/Output/Graphs/Weights_series} \end{center}
\end{frame}

\begin{frame}{Comparing performance of weighted gap - Full Sample}
\protect\hypertarget{comparing-performance-of-weighted-gap---full-sample}{}
\resizebox{\linewidth}{!}{
\begin{tabular}[t]{lrrr>{}rrrrr}
\toprule
Cycles & BIC & AIC & AUC & psAUC & c.Threshold & Type.I & Type.II & Policy.Loss.Function\\
\midrule
null & 0.0000 & 0.0000 & 0.5000 & \textbf{0.5000} &  & 1.0000 & 0.0000 & 1.0000\\
\textbf{1.sided weighted.cycle} & \textbf{-127.5308} & \textbf{-133.9135} & \textbf{0.7182} & \textbf{\textbf{0.6454}} & \textbf{2.8892} & \textbf{0.3532} & \textbf{0.3255} & \textbf{0.2307}\\
c.bn6.r20 & -108.0679 & -114.4506 & 0.7048 & \textbf{0.6379} & 0.6581 & 0.3962 & 0.3019 & 0.2481\\
c.hamilton28.panel & -149.8518 & -156.2346 & 0.7107 & \textbf{0.6359} & 9.7674 & 0.3912 & 0.3066 & 0.2470\\
c.ma & -120.8108 & -127.1936 & 0.6922 & \textbf{0.6313} & 5.7813 & 0.3989 & 0.3160 & 0.2590\\
\addlinespace
c.hamilton13.panelr15 & -126.2968 & -132.6796 & 0.6924 & \textbf{0.6311} & 6.5289 & 0.4297 & 0.2830 & 0.2647\\
c.hamilton28.panelr20 & -164.6015 & -170.9842 & 0.7158 & \textbf{0.6302} & 10.8558 & 0.3948 & 0.2925 & 0.2414\\
c.hamilton28.panelr15 & -154.4533 & -160.8361 & 0.7091 & \textbf{0.6270} & 11.5510 & 0.3854 & 0.2972 & 0.2369\\
c.hamilton13.panel & -133.9347 & -140.3175 & 0.6922 & \textbf{0.6250} & 4.9769 & 0.4285 & 0.2877 & 0.2664\\
c.bn2.r20 & -109.3128 & -115.6955 & 0.6963 & \textbf{0.6218} & 0.2776 & 0.4080 & 0.3255 & 0.2724\\
\addlinespace
c.linear & -135.4069 & -141.7896 & 0.6879 & \textbf{0.6204} & 3.9989 & 0.4616 & 0.2925 & 0.2986\\
c.bn6 & -132.7915 & -139.1742 & 0.6835 & \textbf{0.6113} & 0.4710 & 0.4371 & 0.2830 & 0.2712\\
c.bn2.r15 & -83.9469 & -90.3297 & 0.6749 & \textbf{0.6047} & 0.1349 & 0.4761 & 0.3302 & 0.3357\\
c.poly4.r20 & 3.5738 & -2.8090 & 0.5772 & \textbf{0.6011} & 0.1651 & 0.4980 & 0.3302 & 0.3570\\
\textbf{BIS Basel gap} & \textbf{-121.5910} & \textbf{-127.9738} & \textbf{0.6733} & \textbf{\textbf{0.5960}} & \textbf{3.0578} & \textbf{0.4441} & \textbf{0.3255} & \textbf{0.3032}\\
\addlinespace
c.bn4 & -169.1186 & -175.5014 & 0.6892 & \textbf{0.5943} & 1.2840 & 0.3837 & 0.3255 & 0.2532\\
c.stm.r15 & -79.5531 & -85.9358 & 0.6575 & \textbf{0.5924} & 2.0027 & 0.4778 & 0.3160 & 0.3281\\
c.hp125k & -92.2897 & -98.6725 & 0.6562 & \textbf{0.5924} & 2.5216 & 0.4547 & 0.3302 & 0.3158\\
c.hp221k & -106.8842 & -113.2670 & 0.6656 & \textbf{0.5921} & 2.6641 & 0.4561 & 0.3160 & 0.3079\\
c.hp400k.r15 & -67.1228 & -73.5055 & 0.6472 & \textbf{0.5912} & 2.6223 & 0.4592 & 0.3255 & 0.3168\\
\addlinespace
c.stm & -89.2228 & -95.6055 & 0.6523 & \textbf{0.5903} & 2.2064 & 0.4684 & 0.3302 & 0.3284\\
c.bn3.r15 & -144.4817 & -150.8645 & 0.6687 & \textbf{0.5882} & 0.1862 & 0.4780 & 0.3302 & 0.3375\\
c.hp400k.r20 & -88.8450 & -95.2277 & 0.6545 & \textbf{0.5871} & 2.8130 & 0.4494 & 0.3302 & 0.3110\\
c.stm.r20 & -87.2179 & -93.6006 & 0.6482 & \textbf{0.5859} & 1.9362 & 0.4826 & 0.3302 & 0.3419\\
c.hp25k.r15 & -55.8805 & -62.2632 & 0.6275 & \textbf{0.5812} & 1.1403 & 0.5032 & 0.3066 & 0.3473\\
\addlinespace
c.hp25k & -56.0388 & -62.4215 & 0.6274 & \textbf{0.5782} & 1.2839 & 0.4970 & 0.3160 & 0.3469\\
\bottomrule
\end{tabular}}
\end{frame}

\begin{frame}{Comparing performance of weighted gap - Full Sample}
\protect\hypertarget{comparing-performance-of-weighted-gap---full-sample-1}{}
\begin{block}{Out-of-sample prediction}
\protect\hypertarget{out-of-sample-prediction}{}
\resizebox{\linewidth}{!}{
\begin{tabular}[t]{lll>{}lllll}
\toprule
Cycle & BIC & AUC & psAUC & c.Threshold & Type I & Type II & Policy Loss Function\\
\midrule
\cellcolor{gray!6}{c.null} & \cellcolor{gray!6}{0.0000} & \cellcolor{gray!6}{0.4767} & \textbf{\cellcolor{gray!6}{0.4907}} & \cellcolor{gray!6}{0.0000} & \cellcolor{gray!6}{1.0000} & \cellcolor{gray!6}{0.0000} & \cellcolor{gray!6}{1.0000}\\
 & (1.8172) & (0.0128) & \textbf{(0.0049)} &  &  &  & \\
\addlinespace
\textbf{\cellcolor{gray!6}{1.sided weighted.cycle}} & \textbf{\cellcolor{gray!6}{-80.9638}} & \textbf{\cellcolor{gray!6}{0.7134}} & \textbf{\textbf{\cellcolor{gray!6}{0.6404}}} & \textbf{\cellcolor{gray!6}{2.9939}} & \textbf{\cellcolor{gray!6}{0.3707}} & \textbf{\cellcolor{gray!6}{0.3208}} & \textbf{\cellcolor{gray!6}{0.2411}}\\
\textbf{} & \textbf{(5.0654)} & \textbf{(0.0033)} & \textbf{\textbf{(0.0037)}} & \textbf{(0.6513)} & \textbf{(0.0282)} & \textbf{(0.0092)} & \textbf{(0.0164)}\\
\addlinespace
\cellcolor{gray!6}{c.hamilton28.panel} & \cellcolor{gray!6}{-121.3928} & \cellcolor{gray!6}{0.7078} & \textbf{\cellcolor{gray!6}{0.6318}} & \cellcolor{gray!6}{9.6472} & \cellcolor{gray!6}{0.4043} & \cellcolor{gray!6}{0.3071} & \cellcolor{gray!6}{0.2583}\\
 & (3.0642) & (0.0018) & \textbf{(0.0027)} & (1.1148) & (0.0119) & (0.0165) & (0.0070)\\
\addlinespace
\cellcolor{gray!6}{c.hamilton13.panel} & \cellcolor{gray!6}{-123.8627} & \cellcolor{gray!6}{0.6894} & \textbf{\cellcolor{gray!6}{0.6220}} & \cellcolor{gray!6}{5.1997} & \cellcolor{gray!6}{0.4206} & \cellcolor{gray!6}{0.3132} & \cellcolor{gray!6}{0.2757}\\
 & (4.9205) & (0.0021) & \textbf{(0.0034)} & (0.6877) & (0.0194) & (0.0199) & (0.0092)\\
\addlinespace
\cellcolor{gray!6}{c.hamilton28.panelr15} & \cellcolor{gray!6}{-132.7462} & \cellcolor{gray!6}{0.7052} & \textbf{\cellcolor{gray!6}{0.6217}} & \cellcolor{gray!6}{11.6613} & \cellcolor{gray!6}{0.3874} & \cellcolor{gray!6}{0.3127} & \cellcolor{gray!6}{0.2486}\\
 & (5.5529) & (0.0041) & \textbf{(0.0053)} & (0.5448) & (0.0245) & (0.0135) & (0.0141)\\
\addlinespace
\cellcolor{gray!6}{c.linear} & \cellcolor{gray!6}{-106.5647} & \cellcolor{gray!6}{0.6846} & \textbf{\cellcolor{gray!6}{0.6158}} & \cellcolor{gray!6}{4.0160} & \cellcolor{gray!6}{0.4621} & \cellcolor{gray!6}{0.3113} & \cellcolor{gray!6}{0.3108}\\
 & (4.8347) & (0.0017) & \textbf{(0.0028)} & (1.5341) & (0.0110) & (0.0156) & (0.0080)\\
\addlinespace
\cellcolor{gray!6}{c.bn2.r20} & \cellcolor{gray!6}{-76.7863} & \cellcolor{gray!6}{0.6908} & \textbf{\cellcolor{gray!6}{0.6137}} & \cellcolor{gray!6}{0.2458} & \cellcolor{gray!6}{0.4327} & \cellcolor{gray!6}{0.3208} & \cellcolor{gray!6}{0.2907}\\
 & (2.1205) & (0.0043) & \textbf{(0.0066)} & (0.2248) & (0.0239) & (0.0116) & (0.0149)\\
\addlinespace
\cellcolor{gray!6}{c.bn6.r20} & \cellcolor{gray!6}{-86.4636} & \cellcolor{gray!6}{0.6757} & \textbf{\cellcolor{gray!6}{0.5987}} & \cellcolor{gray!6}{0.2144} & \cellcolor{gray!6}{0.4713} & \cellcolor{gray!6}{0.3231} & \cellcolor{gray!6}{0.3286}\\
 & (2.8925) & (0.0160) & \textbf{(0.0205)} & (1.1466) & (0.0471) & (0.0114) & (0.0445)\\
\addlinespace
\cellcolor{gray!6}{c.bn6} & \cellcolor{gray!6}{-101.2737} & \cellcolor{gray!6}{0.6693} & \textbf{\cellcolor{gray!6}{0.5935}} & \cellcolor{gray!6}{0.3956} & \cellcolor{gray!6}{0.4611} & \cellcolor{gray!6}{0.3250} & \cellcolor{gray!6}{0.3203}\\
 & (4.0553) & (0.0130) & \textbf{(0.0176)} & (0.6125) & (0.0472) & (0.0061) & (0.0459)\\
\addlinespace
\textbf{\cellcolor{gray!6}{BIS Basel gap}} & \textbf{\cellcolor{gray!6}{-110.9026}} & \textbf{\cellcolor{gray!6}{0.6707}} & \textbf{\textbf{\cellcolor{gray!6}{0.5932}}} & \textbf{\cellcolor{gray!6}{3.3012}} & \textbf{\cellcolor{gray!6}{0.4490}} & \textbf{\cellcolor{gray!6}{0.3250}} & \textbf{\cellcolor{gray!6}{0.3073}}\\
\textbf{} & \textbf{(4.6247)} & \textbf{(0.0018)} & \textbf{\textbf{(0.0025)}} & \textbf{(0.7148)} & \textbf{(0.0080)} & \textbf{(0.0052)} & \textbf{(0.0059)}\\
\bottomrule
\multicolumn{8}{l}{\rule{0pt}{1em}\textit{Note: }}\\
\multicolumn{8}{l}{\rule{0pt}{1em}3-fold cross-validation results. Standard deviations are reported in parentheses.}\\
\end{tabular}}
\end{block}
\end{frame}

\begin{frame}{Comparing performance of weighted gap as an EWI - EME}
\protect\hypertarget{comparing-performance-of-weighted-gap-as-an-ewi---eme}{}
\resizebox{\linewidth}{!}{
\begin{tabular}[t]{lrrr>{}rrrrr}
\toprule
Cycles & BIC & AIC & AUC & psAUC & c.Threshold & Type.I & Type.II & Policy.Loss.Function\\
\midrule
null & 0.0000 & 0.0000 & 0.5000 & \textbf{0.5000} &  & 1.0000 & 0.0000 & 1.0000\\
c.bn3.r15 & -46.2774 & -51.3507 & 0.7365 & \textbf{0.6308} & 0.6244 & 0.3059 & 0.3333 & 0.2047\\
c.poly3 & 5.3862 & 0.3129 & 0.5737 & \textbf{0.6046} & 1.8089 & 0.5280 & 0.3056 & 0.3721\\
c.bn2.r15 & -13.2062 & -18.2795 & 0.6879 & \textbf{0.5879} & 0.2952 & 0.3566 & 0.3333 & 0.2383\\
c.poly4.r20 & 7.0732 & 1.9999 & 0.5040 & \textbf{0.5816} & -0.9609 & 0.5962 & 0.3333 & 0.4665\\
\addlinespace
\textbf{1.sided weighted.cycle} & \textbf{6.2094} & \textbf{1.1361} & \textbf{0.5325} & \textbf{\textbf{0.5811}} & \textbf{-1.0639} & \textbf{0.6827} & \textbf{0.1111} & \textbf{0.4784}\\
c.linear & -9.3676 & -14.4409 & 0.5787 & \textbf{0.5774} & -0.9783 & 0.6294 & 0.2222 & 0.4455\\
c.bn2.r20 & -16.6411 & -21.7144 & 0.6760 & \textbf{0.5751} & 0.1470 & 0.4510 & 0.3056 & 0.2968\\
c.hamilton13 & 6.5749 & 1.5016 & 0.5468 & \textbf{0.5710} & 3.6354 & 0.6206 & 0.3056 & 0.4785\\
c.ma & -11.6401 & -16.7133 & 0.5572 & \textbf{0.5457} & -0.2250 & 0.7220 & 0.1667 & 0.5491\\
\addlinespace
c.hamilton28.panel & -7.8687 & -12.9420 & 0.5392 & \textbf{0.5384} & -1.7750 & 0.6958 & 0.2778 & 0.5613\\
c.quad & 6.1997 & 1.1264 & 0.4654 & \textbf{0.5334} & -6.4882 & 0.7456 & 0.1944 & 0.5938\\
c.hamilton13.panelr15 & -1.6660 & -6.7393 & 0.5087 & \textbf{0.5274} & -1.1064 & 0.7002 & 0.3333 & 0.6014\\
c.hp25k.r15 & 3.6420 & -1.4313 & 0.5018 & \textbf{0.5265} & -3.5672 & 0.7850 & 0.1111 & 0.6285\\
c.hp25k & 3.9466 & -1.1267 & 0.4975 & \textbf{0.5247} & -3.7339 & 0.7893 & 0.1111 & 0.6354\\
\addlinespace
c.hp3k & 3.3678 & -1.7054 & 0.5276 & \textbf{0.5235} & -1.1119 & 0.7019 & 0.3333 & 0.6038\\
c.hp3k.r20 & 3.3703 & -1.7030 & 0.5276 & \textbf{0.5235} & -1.1125 & 0.7028 & 0.3333 & 0.6050\\
c.hamilton13.panel & -1.6294 & -6.7027 & 0.5166 & \textbf{0.5222} & -2.9398 & 0.7500 & 0.2778 & 0.6397\\
\textbf{BIS Basel gap} & \textbf{-0.7015} & \textbf{-5.7748} & \textbf{0.4928} & \textbf{\textbf{0.5217}} & \textbf{-5.3969} & \textbf{0.7920} & \textbf{0.1389} & \textbf{0.6465}\\
c.hamilton28.panelr20 & -4.9986 & -10.0719 & 0.5123 & \textbf{0.5213} & -1.5578 & 0.6932 & 0.3333 & 0.5916\\
\addlinespace
c.hamilton28.panelr15 & -3.3914 & -8.4647 & 0.4987 & \textbf{0.5162} & -1.8326 & 0.7220 & 0.3333 & 0.6324\\
c.hp400k.r15 & 5.0603 & -0.0129 & 0.4777 & \textbf{0.5147} & -5.7358 & 0.8121 & 0.1111 & 0.6718\\
c.stm.r15 & 4.7567 & -0.3166 & 0.4780 & \textbf{0.5129} & -5.6472 & 0.8191 & 0.0833 & 0.6778\\
c.hp221k & 1.5902 & -3.4831 & 0.4787 & \textbf{0.5123} & -5.6416 & 0.8121 & 0.1111 & 0.6718\\
c.stm.r20 & 3.1397 & -1.9336 & 0.4768 & \textbf{0.5095} & -5.3727 & 0.8226 & 0.0833 & 0.6835\\
\addlinespace
c.hp125k & 3.0774 & -1.9958 & 0.4740 & \textbf{0.5094} & -5.8918 & 0.8226 & 0.0833 & 0.6835\\
\bottomrule
\end{tabular}}
\end{frame}

\begin{frame}{Plot weighted gap against BIS gap}
\protect\hypertarget{plot-weighted-gap-against-bis-gap}{}
\begin{center}\includegraphics[width=1\linewidth]{../../3rdPaper/Data/Output/Graphs/Weighted_credit_gap_US} \end{center}
\end{frame}

\begin{frame}{Plot weighted gap against BIS gap}
\protect\hypertarget{plot-weighted-gap-against-bis-gap-1}{}
\begin{center}\includegraphics[width=1\linewidth]{../../3rdPaper/Data/Output/Graphs/Weighted_credit_gap_UK} \end{center}
\end{frame}

\begin{frame}{Out-of-sample Prediction for Individual Countriess}
\protect\hypertarget{out-of-sample-prediction-for-individual-countriess}{}
We extrapolated our one-sided weight vector from 2017:Q4 to 2021:Q3, and
analyzed the weighted gap as an EWI. Our model identified 9 countries
that are experiencing pre-crisis periods:

\begin{itemize}
\tightlist
\item
  Canada, France, Hong Kong (SAR), Japan, South Korea, Saudi Arabia,
  Switzerland, Sweden, and Thailand
\end{itemize}
\end{frame}

\begin{frame}{Thank You}
\protect\hypertarget{thank-you}{}
\begin{block}{I look forward to your questions and comments}
\protect\hypertarget{i-look-forward-to-your-questions-and-comments}{}
\end{block}
\end{frame}

\end{document}
